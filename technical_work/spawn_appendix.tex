\section{Código para realizar la integración COMPSs+OmpSs-2}
\label{appendix:integration}

\begin{lstlisting} [caption={Código necesario para la gestión del modo librería de Nanos6 en nio\_worker\_c.cc y worker\_c.cc},captionpos=b, label={lst:managementnanosworkers}, language=C++]

#ifdef OMPSS2_ENABLED
#include <pthread.h>
#include <nanos6/bootstrap.h>
#include <nanos6/library-mode.h>
#endif

    ...

#ifdef OMPSS2_ENABLED
    char const *error = nanos6_library_mode_init();
    if (error != NULL)
    {
        fprintf(stderr, "Error initializing Nanos6: %s\n", error);
        return 1;
    }
    std::cout << "[C-BINDING] Nanos6 initialized" << std::endl;
#endif

    ...
    
#ifdef OMPSS2_ENABLED
    nanos6_shutdown();
    std::cout << "[C-BINDING] Nanos6 shutdown" << std::endl;
#endif

\end{lstlisting}

\bigskip

\begin{lstlisting} [caption={Función para generar la definición de los structs.},captionpos=b, label={lst:structs}, language=C++]
static void generate_struct_nanos6_wrapper(FILE *outFile, 
Types current_types,
function *func) {
argument *arg;
fprintf(outFile, "typedef struct {\n", func->methodname);

char* return_type = construct_returntype(func);

if (func->return_type != void_dt && return_type != NULL) {
fprintf(outFile, "\t%s ret;\n", return_type);
}

arg = func->first_argument;
while (arg != NULL) {
fprintf(outFile, "\t%s;\n", construct_type_and_name(arg));
arg = arg->next_argument;
}
fprintf(outFile, "} %s_struct_t;\n", func->methodname);
}
\end{lstlisting}

\bigskip

\begin{lstlisting} [caption={Función para generar el wrapper de la tarea.},captionpos=b, label={lst:wrapper}, language=C++]

static void generate_nanos6_wrapper(FILE *outFile, function* func) {
fprintf(outFile, "void %s_wrapper(void* args) {\n", 
func->methodname);

fprintf(outFile, 
"\t%s_struct_t* struct_ = (%s_struct_t*) args;\n", 
func->methodname, 
func->methodname);

char* func_to_execute;
int printed_chars = 0;

if (func->return_type != NULL && func->return_type != void_dt) {
fprintf(outFile, "\tstruct_->ret = ", func->return_type);
}

if ((func->classname != NULL) && (func->access_static == 0)) {
printed_chars = asprintf(&func_to_execute, 
"\t%s->%s(", 
func->name, 
func->methodname);
} else {
printed_chars = asprintf(&func_to_execute, 
"\t%s(", func->name);
}

if (printed_chars < 0) {
asprintf_error(func_to_execute, 
"ERROR: Not possible to 
generate method execution.\n");
}

fprintf(outFile, "%s", func_to_execute);

argument* args = func->first_argument;

int first = 1;
while (args != NULL) {
if (first) {
first = 0;
}
else {
fprintf(outFile, ", ");
}
fprintf(outFile, "struct_->%s", args->name);

args = args->next_argument;
}

fprintf(outFile, ");\n");
fprintf(outFile, "}\n");
}  
\end{lstlisting}

\bigskip

\begin{lstlisting} [caption={Código para hacer un spawn de una tarea para OmpSs-2 en COMPSs.},captionpos=b, label={lst:spawncompss}, language=C++]
fprintf(outFile, "\t\t\t%s_struct_t %s_struct;\n", func->methodname, 
func->methodname);

assign_arguments_to_struct(outFile, func);

char* nanos6_spawner;
printed_chars = asprintf(&nanos6_spawner, 
"\t\t\tnanos6_spawn_function(%s_wrapper,
&%s_struct, 
%s, %s, 
\"%s_spawned_task\");\n",
func->methodname, 
func->methodname, 
"condition_variable_callback", 
"&cond_var", 
func->methodname);

if (printed_chars < 0) {
asprintf_error(nanos6_spawner, 
"Not possible to generate function spawn.\n");
}

fprintf(outFile, "%s\n", nanos6_spawner);

fprintf(outFile, "\t\t\twait_condition_variable(&cond_var);\n");

if (func->return_type != void_dt && func->return_type != NULL) {
char* return_var;
printed_chars = asprintf(&return_var, 
"return_obj = %s_struct.ret;\n", 
func->methodname);

fprintf(outFile, "%s", return_var);

if (printed_chars < 0) {
asprintf_error(return_var, 
"Not possible to generate return variable.\n");
}
}
\end{lstlisting}
