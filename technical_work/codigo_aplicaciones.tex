\chapter{Código aplicaciones}
En este apéndice se incluye el código principal y la interfaz de las aplicaciones de \textit{COMPSs} que se utilizan para evaluar la integración con \textit{OmpSs-2}.

\section{K-Means}
\label{sec:codigokmeans}

%\begin{minipage}{\linewidth}
\begin{lstlisting} [caption={Programa principal de la aplicación K-Means.},captionpos=b, 
label={lst:mpkmeans}, language=C++]

void kmeans(int numClusters, 
			int numFrags, 
			int objsFrag, 
			int numCoords, 
			int loop_iteration, 
			char* filePath, 
			int isBinaryFile, 
			int is_output_timing)
{
    printf("Execution: K = % d, 
    		Frags = % d, NpF = % d, 
    		d = % d, max_iters = % d\n", 
		    numClusters, numFrags, objsFrag, numCoords, loop_iteration);
 
    int  i, j, index, loop=1;
    double  timing, io_timing, clustering_timing;

    if (numCoords<1){
        fprintf(stderr,"Error reading number of coordinates");
        exit(-1);
    }
    srand(1000);

    float *clusters = (float*) malloc(numClusters*numCoords*sizeof(float));
    for (i=0; i<numClusters; i++){
        for (j=0; j<numCoords; j++){
            clusters[i*numCoords+j] = (float) ((rand()) / (float)((RAND_MAX))*2 - 1);
        }
    
    int **newClusterSize = (int **) malloc(numFrags*sizeof(int*));
    float **newClusters = (float **) malloc(numFrags*sizeof(float*));
    int **frag_index = (int **)malloc(numFrags*sizeof(int*));
    float **fragments = (float **)malloc(numFrags*sizeof(float*));

    for (j=0; j<numFrags; j++){
        frag_index[j] = (int*) malloc(sizeof(int));
        newClusters[j] = (float*) malloc(numClusters*numCoords*sizeof(float));
    	newClusterSize[j]=(int*) malloc(numClusters* sizeof(int));
        for (i=0; i<numClusters; i++){
		 newClusterSize[j][i]=0;
        }
    }
    compss_on();

    if (is_output_timing) io_timing = wtime();

    for (i=0; i<numFrags; i++){
        fragments[i] = init_Fragment(objsFrag, 
									 numCoords,
									 objsFrag*numCoords,
									 filePath,
									 frag_index[i]);
    }
    compss_barrier();

    if (is_output_timing) {
        timing            = wtime();
        io_timing         = timing - io_timing;
        clustering_timing = timing;
    }

    do {
        for (i=0; i<numFrags; i++){
	    	compute_newCluster(objsFrag, numCoords, 
							   numClusters, objsFrag*numCoords,
							   numClusters*numCoords,
							   frag_index[i], fragments[i],
							   clusters, newClusters[i],
							   newClusterSize[i]);
        }
        int neighbor = 1;
        while (neighbor < numFrags) {
            for (int f = 0; f < numFrags; f += 2 * neighbor) {
                if (f + neighbor < numFrags) {
        			merge_newCluster(numCoords, numClusters, 
									 numClusters*numCoords, 
									 newClusters[f], 
									 newClusters[f+neighbor],
									 newClusterSize[f], 
									 newClusterSize[f+neighbor]);
                }
            }
            neighbor *= 2;
        }
        update_Clusters(numCoords, numClusters, 
						numClusters*numCoords,
						clusters, newClusters[0],
						newClusterSize[0]);

    } while (loop++ < loop_iteration);
    compss_barrier();

    if (is_output_timing) {
        timing            = wtime();
        clustering_timing = timing - clustering_timing;
        printf("\nPerforming **** Regular Kmeans (compss version) ****\n");
        printf("Input file:     % s\n", filePath);
        printf("numFrags     =  % d\n", numFrags);
        printf("numClusters   = % d\n", numClusters);
        printf("Loop iterations    = % d\n", loop_iteration);
        printf("I/O time           = % 10.4f sec\n", io_timing);
        printf("Computation timing = % 10.4f sec\n", clustering_timing);
    }
    compss_off();
}
\end{lstlisting}
%\end{minipage}

\begin{minipage}{\linewidth}
\begin{lstlisting} [caption={Programa principal de la aplicación Cholesky.},captionpos=b, 
label={lst:idlkmeans},style=idlstyle]
interface kmeans{

        @Constraints(ComputingUnits=5);
        float[size] init_Fragment(in int numCoords,
								  in int numObjs,
								  in int size,
								  in File filename,
								  out int[1] ff);

        @Constraints(ComputingUnits=150);
        void compute_newCluster(in int objsFrag, in int numCoords, 
								in int numClusters, in int sizeFrags, 
								in int sizeClusters, in int[1] ff, 
								in float[sizeFrags] frag, 
								in float[sizeClusters] clusters, 
								out float[sizeClusters] newClusters, 
								out int[numClusters] newClustersSize);

        @Constraints(processors={@Processor(ProcessorType=GPU, 
											ComputingUnits=1)});
        @Implements(compute_newCluster);
        void compute_newCluster_GPU(in int objsFrag, in int numCoords,
									in int numClusters, in int sizeFrags,
									in int sizeClusters, in int[1] ff,
									in float[sizeFrags] frag,
									in float[sizeClusters] clusters,
									out float[sizeClusters] newClusters, 
									out int[numClusters] newClustersSize);

       void merge_newCluster(in int numCoords, in int numClusters, 
							 in int sizeClusters, 
							 inout float[sizeClusters] clusters1, 
							 in float[sizeClusters] clusters2,
							 inout int[numClusters] newClustersSize1, 
							 in int[numClusters] newClustersSize2);

       void update_Clusters(in int numCoords, in int numClusters,
							in int sizeClusters, 
							inout float[sizeClusters] clusters1,
							in float[sizeClusters] clusters2,
							in int[numClusters] newClustersSize2);
};

\end{lstlisting}
\end{minipage}

\newpage

\section{Cholesky}
\label{sec:codigocholesky}

%\begin{minipage}{\linewidth}
\begin{lstlisting} [caption={Programa principal de la aplicación Cholesky.},captionpos=b, 
label={lst:mpcholesky}, language=C++]
int main(int argc, char** argv) {

    int MSIZE = atoi(argv[1]);
    int BSIZE = atoi(argv[2]);

    compss_on();
    
    struct timeval start_io, end_io, start_comp, end_comp;
    double io_timing, cholesky_timing;
    
    gettimeofday(&start_io, NULL);

    double** MATRIX = (double**)malloc(sizeof(double*)*MSIZE*MSIZE);
    
    for (int i = 0; i < MSIZE*MSIZE; ++i) {
      MATRIX[i] = (double*)malloc(sizeof(double)*BSIZE*BSIZE);
    }   
 
   for (int i = 0; i < MSIZE; i++) {
      generate_block(MATRIX[i*MSIZE+i], BSIZE, 1);
      for (int j = i+1; j < MSIZE; j++) {
            generate_block(MATRIX[i*MSIZE+j], BSIZE, 0);
            generate_block(MATRIX[j*MSIZE+i], BSIZE, 0);
        }
    } 

    compss_barrier();

    gettimeofday(&end_io, NULL);

    io_timing = (end_io.tv_sec - start_io.tv_sec) * 1e6;
    io_timing = (io_timing + (end_io.tv_usec - start_io.tv_usec)) * 1e-6; 

    gettimeofday(&start_comp, NULL);

    for (int k = 0; k < MSIZE; ++k) {
        ddss_dpotrf(Lower, BSIZE, MATRIX[k*MSIZE+k], BSIZE);

        for (int i = k+1; i < MSIZE; ++i) {
            ddss_dtrsm(Right, Lower, 
		               Trans, NonUnit, 
                	   BSIZE, BSIZE,
                       BSIZE, MATRIX[k*MSIZE+k], BSIZE,
                       MATRIX[i*MSIZE+k], BSIZE);
        }        

        for (int i = k+1; i < MSIZE; ++i) {

            ddss_dsyrk(Lower, NoTrans,
                       BSIZE, BSIZE,
                       BSIZE, MATRIX[i*MSIZE+k], BSIZE,
                      -BSIZE, MATRIX[i*MSIZE+i], BSIZE);
        
            for (int j = i; j < MSIZE; ++j) {

                ddss_dgemm(NoTrans, Trans,
                    	   BSIZE, BSIZE, BSIZE,
                           BSIZE, MATRIX[i*MSIZE+k], BSIZE,
                           MATRIX[j*MSIZE+k], BSIZE,
                           -BSIZE, MATRIX[i*MSIZE+j], BSIZE);
            }
        }
    }

    compss_barrier();

    gettimeofday(&end_comp, NULL);

    cholesky_timing = (end_comp.tv_sec - start_comp.tv_sec) * 1e6;
    cholesky_timing = (cholesky_timing + 
					   (end_comp.tv_usec - start_comp.tv_usec)) * 1e-6; 

    printf("\nPerforming **** Cholesky ****\n");
    printf("I/O time           = %10.4f sec\n", io_timing);
    printf("Computation timing = %10.4f sec\n", cholesky_timing);

    compss_off(); 
}
\end{lstlisting}
%\end{minipage}

\begin{lstlisting} [caption={Interfaz de la aplicación Cholesky.},captionpos=b, 
label={lst:idlcholesky}, style=idlstyle]
include lass.h;
  
interface cholesky
{

    @Constraints(ComputingUnits=2);
    void generate_block(inout double[N*N] A, in int N, in int D);

    @Constraints(ComputingUnits=4);
    int ddss_dpotrf(in enum DDSS_UPLO UPLO, 
					in int N, 
					inout double[N*LDA] A, 
					in int LDA);

    @Constraints(ComputingUnits=4);
    int ddss_dtrsm( in enum DDSS_SIDE SIDE, in enum DDSS_UPLO UPLO,
                    in enum DDSS_TRANS TRANS_A, 
					in enum DDSS_DIAG DIAG,
                    in int M, in int N,
                    in double ALPHA, in double[N*N] A, in int LDA,
                    inout double[M*N] B, in int LDB);

    @Constraints(ComputingUnits=4);
    int ddss_dgemm( in enum DDSS_TRANS TRANS_A, 
					in enum DDSS_TRANS TRANS_B,
                    in int M, in int N, in int K,
                    in double ALPHA, in    double[M*K] A, in int LDA,
                    in double[K*N] B, in int LDB,
                    in double BETA,  inout double[M*N] C, in int LDC );

    @Constraints(ComputingUnits=4);
    int ddss_dsyrk( in enum DDSS_UPLO UPLO, in enum DDSS_TRANS TRANS,
                    in int N, in int K,
                    in double ALPHA, in    double[N*K] A, 
					in int LDA, in double BETA,  
					inout double[N*N] C, in int LDC );

};
\end{lstlisting}
