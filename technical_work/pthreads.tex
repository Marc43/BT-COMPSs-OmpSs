\externaldocument{technical_work/ompss-2-integration.tex}

\section{Pthreads}
\label{appendix:pthread}

En este apéndice se introduce superficialmente qué es un \textit{Pthread} y se hace hincapié en el mecanismo utilizado para la sincronización con las tareas registradas en \textit{Nanos6} con el modo librería introducido en la sección \ref{spawnfunction} \todo{quitar esto, y explicar algo mas sobre pthreads}
\bigskip

Históricamente los \textit{threads} se han implementado por cada fabricante o empresa, de una manera específica, complicando la portabilidad entre plataformas. Los \textit{Pthreads} nos proveen un estándar con la intención de que sea utilizado por la mayoría de máquinas. En todos los sistemas operativos del tipo \textit{UNIX} se implementan los \textit{threads} con el estándar \textit{IEEE POSIX 1003.1c}, y se les llama \textit{POSIX threads} o \textit{Pthreads}. El manual se puede encontrar en internet, y explica de lo más básico a lo más avanzado \cite{barney2009posix}. \todo{la cita ?}
\bigskip











