
\section{Integración de OmpSs-2 en C/C++ COMPSs}

Con tal de realizar una buena integración necesitamos conocer bien cómo funcionan y/o cómo están hechos los componentes a integrar. Con tal de adquirir los conocimientos necesarios, vamos a indagar en la estructura interna del \textit{binding} de \textit{C/C++} de \textit{COMPSs} y a entender cómo se desarrolla y compila una aplicación. También deberemos ver cómo funciona la \textit{API} del \textit{runtime} de \textit{OmpSs-2} \textit{Nanos6} y el proceso habitual de desarrollo y compilado de una aplicación que utiliza el modo librería.

\subsection{Estructura de los binding}

El \textit{runtime} de \textit{COMPSs} fue desarrollado en \textit{Java}, por lo que si queremos soportar cualquier lenguaje (salvo el propio \textit{Java}), necesitamos de alguna manera establecer comunicación con ese lenguaje. Es decir, necesitaremos un mecanismo que nos permita ejecutar código de este lenguaje a soportar, por supuesto, en ambas direcciones. 
\par\bigskip
Actualmente, \textit{COMPSs} cuenta con los \textit{bindings} de \textit{C/C++} y \textit{Python} (usualmente conocido como \textit{PyCOMPSs}), que utilizan estos mecanismos descritos anteriormente. Para efectuar la ejecución entre \textit{Java} y \textit{C/C++} se utiliza la \textit{Java Native Interface}, dado que desde \textit{Python} se puede utilizar la \textit{Python-C API} para , con un componente intermedio entre \textit{Python} y \textit{Java} escrito en \textit{C/C++} (utilizando la \textit{JNI}) permitiríamos efectuar llamadas desde \textit{C/C++} y \textit{Python} al \textit{runtime} y al revés. 
\par\bigskip
La siguiente imagen muestra la estructura general de los \textit{bindings} actuales. Las cajas representan componentes de la arquitectura, el color de cada caja ha sido escogido para representar un lenguaje, por lo que dos cajas que se comuniquen directamente con un flecha no requieren de mecanismos adicionales.

\begin{figure}[H]
    \centering 
    \caption{Estructura de los bindings de COMPSs.}
    %\includegraphics[width=\textwidth]{sta-masterworker.jpg}
    \includegraphics[scale=0.6]{estructuraBindings.png}
    \label{fig:estructura_bindings}
\end{figure}
