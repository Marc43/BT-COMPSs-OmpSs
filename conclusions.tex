\chapter{Conclusiones}

A lo largo de este proyecto hemos podido ver todo lo relacionado con realizar una integración entre dos modelos de programación, seguramente cada integración sea un mundo, pero con este proyecto hemos aprendido a realizar una y seguramente todo esto sea aplicable en otra integración. En concreto se ha visto cómo atacar el problema, conocer los dos modelos lo suficiente como para saber cómo se tiene que realizar la integración y averiguar exactamente qué papel desarrolla cada uno en esta integración. Además se ha aprendido a desarrollar aplicaciones que hagan uso de la integración \textit{COMPSs+OmpSs-2}, que ha requerido entender y saber utilizar los modelos de programación \textit{COMPSs} y \textit{OmpSs-2}. También hemos realizado una evaluación del rendimiento, conociendo dónde están los límites que ponen las aplicaciones y dónde los que pone la integración. 

\par\bigskip

Se ha iniciado este proyecto con tal de brindar facilidad para el uso de las plataformas distribuidas heterogéneas, y se ha conseguido. Todo el esfuerzo puesto sobre la integración nos ha permitido conseguir gestionar estas plataformas de una manera muy sencilla, no sólo facilita la gestión si no que permite mejorar el rendimiento de las aplicaciones para estas plataformas, tal y como se demuestra en la sección \ref{sec:estudiorend}. Las modificaciones que se han realizado sobre \textit{COMPSs} para realizar la integración han intentado ser mínimas y lo más claras posibles, así cualquier persona que tenga que trabajar con el código pueda ser partícipe sin que sea necesariamente un quebradero de cabeza.

\par\bigskip

Esperamos que esta facilidad que hemos introducido motive a otros investigadores a utilizar nuestro modelo de programación \textit{COMPSs} conjuntamente con \textit{OmpSs-2} e incluso nos lleve a colaborar con ellos, ya que somos los más indicados, tenemos conocimiento pleno acerca de la integración.

\par\bigskip

Como trabajo futuro, sería fantástico integrar también el modelo \textit{OmpSs-2} en \textit{Java} y \textit{Python}, de esta manera lenguajes interpretados podrían gozar de la velocidad de un lenguaje compilado como es \textit{C} y del paralelismo que \textit{OmpSs-2} nos brinda. Sería también interesante comparar la eficiencia energética de aplicaciones desarrolladas para \textit{COMPSs+OmpSs-2} y \textit{COMPSs}, así veríamos que hacer uso de la variedad de recursos de los entornos que utilizamos hoy en día no vale la pena tan sólo en términos de rendimiento si no también en términos de eficiencia energética.
 
\par\bigskip

A nivel personal el proyecto ha complementado la mayoría de conocimientos adquiridos durante el grado, se ha mucho sobre modelos de programación para sistemas distribuidos (\textit{COMPSs}) y paralelos (\textit{OmpSs-2}), se ha tenido que utilizar \textit{Extrae} y \textit{Paraver} a un nivel más avanzado y también se ha aprendido cómo modificar un compilador hecho con \textit{Yacc} y \textit{Lex}. Además me ha permitido iniciarme en el mundo de la investigación gracias a la confianza de mi director y co-director.
