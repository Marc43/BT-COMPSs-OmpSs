\section{Contextualización del proyecto}

Respecto a la planificación ha habido un par de modificaciones, por cómo está implementado OmpSs-2 no se soportan versiones de cuda inferiores a INSERT por lo cual hemos decidido prescindir del clúster MinoTauro y utilizar solamente CTE-Power.

Los costes se ven reducidos, MODIFICAR TABLA.

La planificación definitiva es, INSERTAR NUEVO GANTT. 

Nos encontramos en …

\section{Planificación}

\section{Metodología y rigor}

La metodología propuesta sigue siendo la misma.

\section{Análisis de alternativas}

No se proponen alternativas dado que el trabajo propone hacerlo con una alternativa en concreto (utilizando COMPSs y OmpSs-2). La opción se justifica en la gestión del proyecto realizada.

\section{Integración de conocimientos}

Se utilizan conocimientos de programación a nivel de threads, de programación en entornos distribuidos, de programación en entornos paralelos, también técnicas de debugging (gdb), lenguajes como C, C++, Java y Bash y herramientas para crear programas que interpreten o compilen como son Yacc y Lex.

\section{Identificación de leyes y regulaciones}

???
