\section{Planificación temporal}

Esta sección presenta como se utilizarán los aproximadamente cuatro meses de duración que tiene el proyecto, desde Febrero de 2019 hasta Junio de 2019. Se especificarán las tareas a realizar junto a su durada aproximada, teniendo en cuenta las posibles desviaciones en la realización de estas. Dado que cualquier desviación puede resultar en la alteración de la planificación, se debe tener en cuenta que la especificación que sigue no debe tomarse al pie de la letra, sino que debe reinterpretarse y modificarse siempre que sea necesario. 

\subsection{Especificación de las tareas}

Detallamos a continuación las tareas a realizar.

\subsubsection{GEP - Gestión de proyectos}

La asignatura GEP conforma el primer bloque del proyecto, se debe elaborar cuatro entregables que sinteticen la temática del proyecto, objetivos, como se realizará (metodología), definir las actividades, realizar un estudio de sostenibilidad y finalmente presentar este documento con soporte visual de manera oral ante un tribunal.

\begin{itemize}
 \item \textbf{Elaboración del primer entregable:} En este primer apartado se dará un contexto, el estado del arte del proyecto, los objetivos, requerimentos, riesgos y una metodología para desarrollar el proyecto en sí. La duración aproximada es de unas 24 horas.
 \item \textbf{Elaboración del segundo entregable:} En este apartado se definen las actividades y duración de estas. La duración aproximada es de 8 horas.
 \item \textbf{Elaboración del tercer entregable:} En este apartado se realizará la autoevaluación sobre la sostenibilidad además de un análisis sobre la gestión económica y la sostenibilidad del proyecto. La duración aproximada es de 15 horas.
 \item \textbf{Elaboración del cuarto entregable:} En este último apartado se preparará una presentación oral y se confeccionará el documento final, que serán los tres anteriores revisados y corregidos con la orientación del \textit{feedback} del profesor de GEP, director y codirector. La duración aproximada es de 15 horas.
\end{itemize}

Con estos cuatro apartados se finaliza el GEP. En princpio, la duración estipulada del GEP es de 75 horas, la suma de las horas que hemos creído que durarían es de 62 horas, esta diferencia nos proporcionará cierto margen para corregir y mejorar los entregables. Para realizar esta actividad, se empleará un ordenador, \textit{GitHub} para subir la documentación, \textit{Kile} para redactar el documento en \textit{LaTeX},\textit{Trello} para organizar las actividades en forma de tarjetas, \textit{Gantter} para elaborar el diagrama de \textit{Gantt} y \textit{Google Drive}.

\subsubsection{Uso de la API de Nanos6}

Para poder llevar acabo exitosamente la integración, se necesita entender qué hace y saber utilizar la \textit{API} de \textit{Nanos6}. Requiere mirar documentación e interactuar con los desarrolladores de \textit{Nanos6}. 
\par\bigskip

Queremos aprender a utilizar la llamada \textit{nanos6\_spawn\_function}, que nos permitirá ejecutar una función como tarea. Para poder utilizarla, necesitamos levantar el \textit{runtime} de manera manual en un programa compilado con \textit{gcc} (\textit{GNU C Compiler}) y efectuar la llamada a una función externa compilada con \textit{mcc} (\textit{Mercurium}), ya que estará anotada con pragmas de \textit{OmpSs-2}.
\par\bigskip

El tiempo aproximado para realizar esta tarea es de 30 horas. Los recursos necesarios son un ordenador con un compilador nativo de \textit{C} y \textit{Mercurium} y \textit{Nanos6} instalados. 

\subsubsection{Integrar OmpSs-2 en el binding de C/C++}

La tarea principal que da sentido al proyecto es esta, comprende el estudio y la integración de \textit{OmpSs-2} en el binding de \textit{C/C++}. Requiere del estudio de la estructura interna de \textit{COMPSs} por una banda y del \textit{binding} por otra, con tal de saber dónde se podría inicializar el \textit{runtime} de \textit{Nanos6} y cuándo se debería apagar. También se necesita hacer la llamada a la \textit{API} en el \textit{worker}, cosa que habrá que estudiar también dónde situar.
\par\bigskip

El primer paso consistirá en analizar dónde tendría más sentido que hagamos la gestión del \textit{runtime} de \textit{Nanos6}, y cómo hacerla. Para esto tendremos que dar un repaso al código de \textit{COMPSs} y esclarecer qué hace cada componente de este. 
\par\bigksip

En segundo lugar se deberá implementar toda la gestión. Se deberá también estudiar dónde se debería hacer la llamada a la \textit{API}, y por verificar el funcionamiento de la integración, que será de seguro lo más complejo y lo que más tiempo requerirá.
\par\bigskip

Esta tarea es la que más tiempo nos llevará, ya que será el primer intento serio de integrarlo todo, pero nos aportará conocimiento pleno de como funciona y por lo tanto facilidad para trabajar en futuras ampliaciones. La duración será aproximadamente de 150 horas. Los recursos necesarios son un ordenador con \textit{COMPSs} instalado, un compilador nativo de \textit{C} y \textit{Mercurium} y \textit{Nanos6} instalados.

\subsubsection{Estudiar la integración de OmpSs-2 en Java y binding de Python}

En caso de que la primera integración haya funcionado, se estudiará la posibilidad de hacer lo mismo con \textit{Java} y el \textit{binding} de \textit{Python}. Consistirá exactamente de los mismos pasos, y puede ayudar a mejorar la implementación anterior. La duración estimada de esta tarea dependerá de si se decide realizar o no esta actividad. Mínimo se emplearán 15 horas en el estudio preliminar, y en caso de realizar la integración, 100 horas más, es decir, 15 horas o bien 115 horas. Los recursos necesarios son un ordenador con \textit{COMPSs} instalado, un compilador nativo de \textit{C} y \textit{Mercurium} y \textit{Nanos6} instalados.

\subsubsection{Desarrollo de una aplicación que use COMPSs+OmpSs-2}

En esta tarea se quiere desarrollar una aplicación que haga uso de \textit{COMPSs+OmpSs-2} con tal de estudiar después el rendimiento de la integración. La aplicación... \todo{Especificar alguna aplicacion en concreto?}

\subsubsection{Estudio del rendimiento}

Utilizando la aplicación desarrollada en la tarea anterior estudiaremos el rendimiento de la integración, haremos uso de herramientas del \textit{BSC} como por ejemplo \textit{Extrae} y \textit{Paraver}, desarrolladas por el grupo \textit{Performance Tools} que respectivamente nos permitirán extraer trazas una vez la aplicación haya acabado y visualizarlas. 
\par\bigskip

Estudiar el rendimiento incluirá intentar optimizar al máximo todas las pérdidas de rendimiento en la medida de lo posible, por lo cual el tiempo aproximado para llevarla realizarla es de 60 horas. Los recursos necesarios son un ordenador con las herramientas mencionadas anteriormente instaladas, \textit{COMPSs} instalado, un compilador nativo de \textit{C} y \textit{Mercurium} y \textit{Nanos6} instalados.

\subsubsection{Redactar la memoria}

Por último se deberá redactar la memoria del proyecto además de preparar todo el material audiovisual para la defensa de este. La duración de esta actividad será de unas 80 horas. Los recursos que se utilizarán son \textit{Kile} para redactar el documento en \textit{LaTeX}, \textit{GitHub} para guardar la documentación y \textit{LibreOffice} para el apoyo audiovisual que se utilizará durante la defensa.

\subsection{Dependencias}

La siguiente tabla define la relación de dependencia entre las tareas que conciernen a la gestión del proyecto.

\begin{center}
 \begin{tabular}{||c c||}
    \hline  
    Tarea Dependiente & Tarea predecesora
    \hline\hline
    Contexto & -
    \hline\hline
    Estado del arte & Contexto
    \hline\hline
    Objetivos, requerimentos, riesgos & Estado del arte
    \hline\hline
    Metodología & Objetivos, requerimentos, riesgos
    \hline\hline
    Definir actividades & Metodología
    \hline\hline
    Estimar tiempos & Definir actividades
    \hline\hline
    Autoevaluación sobre la sostenibilidad & Estimar tiempos
    \hline\hline
    Análisis del proyecto & Autoevaluación sobre la sostenibilidad
    \hline\hline
    Confeccionar documento final & Análisis del proyecto
    \hline\hline
    Preparar presentación & Confeccionar documento final
    \hline\hline
    
    
 \end{tabular}
 \caption{Relación de dependencia para las tareas de la gestión del proyecto.}
\end{center}







