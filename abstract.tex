\afterpage{\blankpage}
\newpage

\noindent
\thispagestyle{plain}
\begin{center}
	\huge{\textbf{Resumen}}
\end{center}

Hoy en día la gran variedad de recursos de cómputo (GPUs, ASICs, FPGAs) puede llegar a ser abrumador y aún más las distintas maneras de programarlos. 
Los programadores que trabajan en entornos heterogéneos tienen que lidiar con más de un modelo y herramientas para conseguir utilizar todos los recursos.
En este proyecto queremos facilitar el uso de entornos distribuidos heterogéneos llevando a cabo la integración de los modelos de programación \textit{COMPSs} y \textit{OmpSs}. \textit{COMP Superscalar} (\textit{COMPSs}) es un modelo de programación para entornos distribuidos basado en la generación de tareas y \textit{Omp Superscalar} (\textit{OmpSs}) intenta explotar el paralelismo de las aplicaciones dentro de un nodo haciendo uso de todos los recursos presentes.
En el proyecto se realiza la integración \textit{COMPSs+OmpSs-2} que permite programar entornos distribuidos heterogéneos de una manera sencilla y se evalúa su rendimiento.

\afterpage{\blankpage}
\newpage

\noindent
\thispagestyle{plain}
\begin{center}
	\huge{\textbf{Resum}}
\end{center}

Avui en dia la gran varietat de recursos de còmput (GPUs, ASICs, FPGAs) pot arribar a ser aclaparadora i encara més les diferents formes de programar-los.
Els programadors que treballen en entorns heterogenis han de lidiar amb més d'un model i eines per tal d'utilitzar tots els recursos. En aquest projecte volem facilitar l'ús dels entorns distribuïts heterogenis fent una integració dels models de programació \textit{COMPSs} i \textit{OmpSs-2}. \textit{COMP Superscalar} (\textit{COMPSs}) es un model de programació per entorns distribuïts basat en la generació de tasques i \textit{Omp Superscalar} (\textit{OmpSs}) intenta explotar el paral·lelisme de les aplicacions dins d'un node fent ús dels recursos d'aquest. En el projecte es realitza la integració \textit{COMPSs+OmpSs-2} que permet programar entorns distribuïts heterogenis de manera senzilla i s'avalua el seu rendiment.

\afterpage{\blankpage}
\newpage

\noindent
\thispagestyle{plain}
\begin{center}
	\huge{\textbf{Abstract}}
\end{center}

Nowadays, variety of computing resources (GPUs, ASICs, FPGAs) can be overwhelming and even more the different ways of programming them. Programmers working in this  distributed heterogeneous environments have to cope with more than a programming model and tools to make profit of these resources. In this project we want to facilitate the use of distributed heterogeneous environments by doing an integration of programming models \textit{COMPSs} and \textit{OmpSs-2}. \textit{COMP Superscalar} (\textit{COMPSs}) is a task-based programming model for distributed environments and \textit{Omp Superscalar} (\textit{OmpSs}) exploits parallelism inside the node using the resources within. In the project the integration \textit{COMPSs+OmpSs-2} is done, it allows the programming of distributed heterogeneous environments in an easy way and a performance evaluation is also done.